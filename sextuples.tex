


\documentclass [11pt] {article}

\usepackage{amsmath,amssymb}

\newtheorem{theorem}{Theorem}
  \newtheorem{lemma}{Lemma}
  \newtheorem{proposition}{Proposition}
  \newtheorem{corollary}{Corollary}
  \newtheorem{definition}{Definition}
  \newtheorem{remark}{Remark}
  \newtheorem{example}{Example}

 \renewcommand{\arraystretch}{1.5}

  \newcommand {\pf}  {\mbox{\sc Proof. \,\,}}

  \newcommand {\qed} {\null \hfill \rule{2mm}{2mm}}
  \def\rank {\mathop{\mathrm{rank}}\nolimits}


\begin{document}

\title{{\Large{\bf Rational Diophantine sextuples with mixed signs}}}

\author{Andrej Dujella}


\date{}
\maketitle


\begin{abstract}
A rational Diophantine $m$-tuple is a set of $m$ nonzero rationals
such that the product of any two of them is one less than a perfect
square. Recently Gibbs constructed several examples of rational
Diophantine sextuples with positive elements. In this note, we
construct examples of rational Diophantine sextuples with mixed
signs. Indeed, we show that such examples exist for all possible
combinations of signs.
\end{abstract}

\footnotetext{
2000 Mathematics Subject Classification: 11D09, 11G05, 11Y50.

Key words: Diophantine sextuples, elliptic curve. }

\section{Introduction}
A set of $m$ nonzero rationals $\{a_1,a_2,\ldots,a_m\}$ is called
{\em a (rational) Diophantine $m$-tuple} if $a_ia_j+1$ is a
perfect square for all $1\leq i<j\leq m$ (see \cite{D-acta2}).

The principal question is how large a rational Diophantine tuple can be.
In case of integer Diophantine tuples, the corresponding question
is almost completely answered. Namely, it is well-known and easy to prove that
there exist infinitely many integer Diophantine quadruples (e.g. $\{k-1,k+1,4k,16k^3-4k\}$
for $k\geq 2$), while it was proved
in \cite{D-crelle} that there does not exist an integer Diophantine sextuple
and there are only finitely many such quintuples (see also \cite{Fuj-reg}).
However, in the case of rational Diophantine tuples, no absolute upper
bound for the size of such sets is known (the existence of a such bound
follows from the Lang conjecture on varieties of general type).
The first example of a rational Diophantine quadruple was the set
$\{\frac{1}{16},\, \frac{33}{16},\, \frac{17}{4},\, \frac{105}{16}\}$
found by Diophantus (see \cite{Dio}).
Euler found infinitely many rational Diophantine quintuples (see \cite{Hea}). E.g.
$$ \left\{ 1, 3, 8, 120, \frac{777480}{8288641} \right\}. $$
Since 1999, several examples of rational Diophantine sextuples were found by Gibbs \cite{Gibbs1,Gibbs2}.
The first example was
$$ \left\{ \frac{11}{192}, \frac{35}{192}, \frac{155}{27}, \frac{512}{27}, \frac{1235}{48},
\frac{180873}{16} \right\} . $$
No example of a rational Diophantine septuple is known.

If $\{a_1,a_2,a_3,a_4,a_5\}$ is a rational Diophantine quintuple, we may consider the hyperelliptic curve
$$ y^2=(a_1x+n)(a_2x+n)(a_3x+n)(a_4x+n)(a_5x+n) $$
of genus $g=2$. Caporaso, Harris and Mazur \cite{C-H-M} proved that
the Lang conjecture on varieties of general type implies that for $g\geq 2$
the number $B(g,\mathbb{K})=\max_{C} |C(\mathbb{K})|$ is finite. Here $C$ runs over all
curves of genus $g$ over a number field $\mathbb{K}$, and $C(\mathbb{K})$ denotes the set of
all $\mathbb{K}$-rational points on $C$. Therefore, the number of elements in a rational
Diophantine tuple should be bounded by $5+B(2,\mathbb{Q})$
(and also by $4+B(4,\mathbb{Q})$, see \cite{P-H-Z}).

\medskip

It can be noted that all Gibbs' examples of sextuples contain six positive rationals.
Thus, it makes sense to ask if there exist such sextuples with mixed signs. Since
$\{a_1, \ldots, a_6\}$ is a Diophantine sextuple if and only if $\{-a_1, \ldots, -a_6\}$ has
the same property, it suffices to find sextuples with exactly one, two and three negative elements.


\section{The constructions}

In the constructions of rational Diophantine sextuples, we use several
techniques. Most of them can be explained in terms of elliptic curves.

If $\{a,b\}$ is a rational Diophantine pair,
then $\{a,b,a+b\pm 2\sqrt{ab+1}\}$ is a rational Diophantine triple.
Such triples are called \emph{regular}.

Let $\{a,b,c\}$ be a (rational) Diophantine triple. In order to extend
this triple to a quadruple, we have to solve the system
\begin{equation} \label{2e}
ax+1=\Box,\qquad bx+1=\Box,\qquad cx+1=\Box.
\end{equation}
It is a natural idea to assign to the system (\ref{2e}) the elliptic curve
\begin{equation} \label{3e}
 \mathcal{E}: \qquad y^2=(ax+1)(bx+1)(cx+1).
\end{equation}
There are three rational points on $E$ of order $2$, and also other obvious rational points
$$ P=[0,1], \quad S=[1, \sqrt{(ab+1)(ac+1)(bc+1)} ]. $$
The $x$-coordinate of a point $T\in E(\mathbb{Q})$ satisfies (\ref{2e}) if and only if
$T-P\in 2\mathcal{E}(\mathbb{Q})$ (see \cite{D-rim}).
It can be verified that $S\in 2\mathcal{E}(\mathbb{Q})$.
This implies that
the numbers $x(P\pm S)$ satisfy the system (\ref{2e}).
These numbers are exactly the numbers
$$ d_{+,-}=a+b+c+2abc\pm 2 \sqrt{(ab+1)(ac+1)(bc+1)} $$
obtained by
Arkin, Hoggatt and Strauss \cite{A-H-S}. They proved that $\{a,b,c,d_{+}\}$ and $\{a,b,c,d_{-}\}$ are
rational Diophantine quadruples (if their elements are distinct and nonzero).
Quadruples of this form are called \emph{regular}. The conjecture is that
all integer Diophantine quadruples are regular.
Note that if $\{a,b,c\}$ is a regular triple, then $d_{+}d_{-}=0$.

Let $\{a,b,c,d\}$ be a rational Diophantine quadruple such that $abcd\neq 1$
and let
\begin{footnotesize}
\[ e_{+,-}=\frac{(a\!+\!b\!+\!c\!+\!d)(abcd+1)+2abc+2abd+2acd+2bcd\pm
2\sqrt{(ab\!+\!1)(ac\!+\!1)(ad\!+\!1)(bc\!+\!1)
(bd\!+\!1)(cd\!+\!1)}} {(abcd-1)^2} \,.\]
\end{footnotesize}
In \cite{D-acta2} we proved that $\{a,b,c,d,e_{+}\}$ and $\{a,b,c,d,e_{-}\}$
are rational Diophantine quintuples (if their elements are distinct and nonzero).
Note that if $\{a,b,c,d\}$ is a regular quadruple, then $e_{+}e_{-}=0$
(see \cite[Proposition 2]{D-acta2}).
A rational Diophantine quintuple $\{a,b,c,d,e\}$ is called {\em regular} if it is
obtained by the construction from \cite{D-acta2} or, equivalently, if
$$ (abcde+2abc+a+b+c-d-e)^2= 4(ab+1)(ac+1)(bc+1)(de+1) \,. $$
This construction can be explained also in the terms of elliptic curve $E$.
Namely, let $D$ be the point on $\mathcal{E}$ with the $x$-coordinate $d$.
Then the numbers $e_{+}$ and $e_{-}$ are exactly the $x$-coordinates of the
points $D\pm S$ on $\mathcal{E}$.
(See \cite{D-irr} for the characterization
of regular quadruples and quintuples in terms of the elliptic curve
$y^2= (ax+1)(bx+1)(cx+1)(dx+1)$.)

\medskip

Now we describe briefly three techniques for contruction of
rational Diophantine sextuples.


\begin{itemize}

\item Let $ab+1=r^2$ and $c=a+b+2r$, i.e. take $\{a,b,c\}$ to be a regular rational
Diophantine triple. Consider the elliptic curve $\mathcal{E}$ given by (\ref{3e}).
This construction uses rational points of relatively small height on $\mathcal{E}$.
For example, if $\rank(E(\mathbb{Q}))=2$ and $X_1,X_2$ are generators
of $E(\mathbb{Q})/E(\mathbb{Q})_{tors}$, we consider the points of the form
$T=m_1X_1+m_2X_2$, for $|m_i| \in \{0,1,2,3 \}$.
If $D=P+2T=[d,d']$, then $\{a,b,c,d\}$ is a rational Diophantine quadruple
(at least some of these quadruples are irregular; this is why we prefere to avoid
curves with rank $1$).
Define the points $E=D+S=[e,e']$ and $E=D-S=[f,f']$.
Then $\{a,b,c,d,e\}$ and $\{a,b,c,d,f\}$ are rational Diophantine quintuples
(and if $\{a,b,c,d\}$ was irregular, then $ef\neq 0$).
If $ef+1$ is a perfect square, then $\{a,b,c,d,e,f\}$ is a rational Diophantine sextuple
(assuming that all its elements are distinct and nonzero).

In that way, we find e.g. the sextuple $\{ \frac{5}{36}, \frac{5}{4}, \frac{32}{9},
\frac{189}{4}, \frac{665}{1521}, \frac{3213}{676} \}$ with positive elements
(found already by Gibbs \cite{Gibbs2}), but also several sextuples with mixed signs, e.g.
$ \{ \frac{5}{14}, \frac{7}{2}, \frac{48}{7}, \frac{1680}{361}, -\frac{2310}{19321}, \frac{93840}{71407} \}$.

\item Take again the regular triple $\{a,b,c\}$, where $c=a+b+2r$, and apply
the same construction to obtain a regular triple $\{b,c,d\}$. We find that $d=a+4b+4r$.
The only remaining condition in order that $\{a,b,c,d\}$ be a Diophantine quadruple
is that $ad+1$ is a perfect square. This condition leads to $(a+2r)^2-3 = \Box$,
and it is satisfied if we take $r=\frac{u-a}{2}$, where $u=\frac{\alpha^2+3}{2\alpha}$
for $\alpha \in \mathbb{Q}$.

Applying the construction from \cite{D-acta2} to the quadruple $\{a,b,c,d\}$,
we obtain the quintuples $\{a,b,c,d,e_{+}\}$ and $\{a,b,c,d,e_{-}\}$.
If $e_{+}e_{-}+1$ is a perfect square, then $\{a,b,c,d,e_{+},e_{-}\}$ is a rational Diophantine sextuple
(assuming that all its elements are distinct and nonzero). As an example of a sextuple obtained
with this construction, we give $\{ \frac{27}{35}, -\frac{35}{36},
\frac{1007}{1260}, -\frac{352}{315}, \frac{72765}{106276}, -\frac{5600}{4489} \}$.

\item Let $\{a,b\}$ be a rational Diophantine pair. For a rational number $t$, define
$c=-\frac{4t(-1+t)(bt-a)}{(-a+bt^2)^2}$. It is easy to check that $ac+1$ and $bc+1$ are
perfect squares, and therefore $\{a,b,c\}$ is a rational Diophantine triple.
We can extend this triple to an (irregular) quadruple by
$d=\frac{8(c-a-b)(a+c-b)(b+c-a)}{(a^2+b^2+c^2-2ab-2ac-2bc)^2}$ (see \cite[Proposition 3]{D-acta2}).
This number is the $x$-coordinate of the point $3P$ on $\mathcal{E}$.
Again, we can apply the construction from \cite{D-acta2} to the quadruple $\{a,b,c,d\}$,
to obtain $e_{+}$ and $e_{-}$, and if $e_{+}e_{-}+1$ is a perfect square,
then we get a rational Diophantine sextuple $\{a,b,c,d,e_{+},e_{-}\}$
(provided that all its elements are distinct and nonzero).
The reason why in this construction we use irregular triples $\{a,b,c\}$ is that for regular
triples, we have $d=d_{+}$, so the resulted quadruple is regular and gives $e_{+}e_{-}=0$.
With this construction we find e.g. the sextuple
$ \{ -\frac{5}{9}, \frac{32}{45}, \frac{27}{20}, \frac{216032}{937445},
-\frac{185232905}{263802564}, \frac{175578975}{136095556} \}$.
\end{itemize}

The described algorithms are implemented in PARI/GP \cite{pari},
and for computing the ranks we use MWRANK \cite{mwrank}.

\section{Examples}

We give the list of $26$ rational Diophantine sextuples with mixed signs
obtained by the constructions described in the previous section (and their slight variations):

{\footnotesize

$$ \Big\{ \frac{19}{12}, \frac{33}{4}, \frac{52}{3}, \frac{60}{2209},
-\frac{495}{24964}, \frac{595}{12} \Big\}, \,\,
 \Big\{ \frac{31}{84}, \frac{9}{7}, \frac{49}{12}, \frac{160}{21},
-\frac{455}{3468}, \frac{7200}{2023} \Big\}, $$
$$ \Big\{ \frac{5}{14}, \frac{7}{2}, \frac{48}{7}, \frac{1680}{361},
-\frac{2310}{19321}, \frac{93840}{71407} \Big\}, \,\,
 \Big\{ \frac{147}{20}, \frac{25}{28}, \frac{96}{35},
-\frac{11}{140}, \frac{30723}{3380}, \frac{165}{1183} \Big\}, $$
$$ \Big\{ -\frac{9}{380}, \frac{253}{1140}, \frac{247}{60},
\frac{125}{57}, \frac{6688}{375}, \frac{2016}{95} \Big\}, \,\,
 \Big\{ \frac{7}{40}, -\frac{75}{56}, \frac{41}{70},
-\frac{5376}{4805}, -\frac{300288}{241115}, \frac{165}{224} \Big\}, $$
$$ \Big\{ \frac{1}{6}, \frac{27}{8}, \frac{385}{96}, \frac{1280}{243},
\frac{250705}{44376}, -\frac{25415}{161376} \Big\}, \,\,
 \Big\{ \frac{27}{35}, -\frac{35}{36}, \frac{1007}{1260},
-\frac{352}{315}, \frac{72765}{106276}, -\frac{5600}{4489} \Big\}, $$
$$ \Big\{ \frac{5}{24}, -\frac{64}{15}, -\frac{407}{120},
-\frac{1530}{361}, \frac{2088}{9245}, \frac{399245}{2889816} \Big\}, \,\,
 \Big\{ -\frac{8}{17}, \frac{85}{72}, -\frac{763}{1224},
\frac{18360}{11449}, \frac{4914}{8993}, \frac{332605}{496008} \Big\}, $$
$$ \Big\{ -\frac{5}{33}, \frac{121}{60}, \frac{131}{660},
\frac{171360}{30899}, \frac{51978528}{54014455}, \frac{8041}{1500} \Big\}, $$
$$ \Big\{ \frac{8}{23}, \frac{161}{72}, \frac{8695}{1656},
\frac{54648}{22201}, -\frac{11270}{62001}, \frac{46288935}{9481336} \Big\}, $$
$$ \Big\{ \frac{27}{14}, \frac{49}{18}, -\frac{16}{63}, \frac{269654}{113569},
\frac{11572496}{19969047}, -\frac{15578784}{44488087} \Big\}, $$
$$ \Big\{ \frac{24}{35}, -\frac{75}{56}, \frac{77}{120}, -\frac{846600}{634207},
\frac{5629624}{7540215}, -\frac{4456963}{3346680}  \Big\}, $$
$$ \Big\{ \frac{5}{9}, -\frac{27}{20}, -\frac{55}{36}, \frac{96305}{158404},
\frac{23144992}{59202405}, -\frac{31157568}{20220605} \Big\}, $$
$$ \Big\{ \frac{5}{9}, -\frac{27}{20}, \frac{13}{20}, -\frac{304083}{212180},
\frac{20055200}{31573161}, -\frac{79520320}{67125249} \Big\}, $$
$$ \Big\{ -\frac{5}{9}, \frac{27}{20}, \frac{32}{45}, \frac{216032}{937445},
-\frac{185232905}{263802564}, \frac{175578975}{136095556} \Big\}, $$
$$ \Big\{ \frac{27}{11}, \frac{77}{36}, -\frac{32}{99}, -\frac{43424}{2297339},
\frac{811864053}{368716804}, -\frac{808311427}{2102956164} \Big\}, $$
$$ \Big\{ \frac{21}{22}, \frac{33}{56}, -\frac{64}{77},
-\frac{3340352}{3625853}, \frac{1092049959}{1018087688}, -\frac{778578801}{1587999368} \Big\}, $$
$$  \Big\{ \frac{27}{35}, -\frac{35}{36}, \frac{161}{180},
-\frac{4771879}{4287380}, \frac{917801280}{4823805007}, -\frac{2117588000}{6213359943} \Big\}, $$
$$ \Big\{ \frac{5}{28}, -\frac{27}{35}, \frac{35}{36}, \frac{3838005}{64606108},
\frac{324705510976}{300303876645}, -\frac{329539009184}{358699363245} \Big\}, $$
$$ \Big\{ \frac{7}{26}, -\frac{221}{72}, -\frac{297}{104},
\frac{226791}{1867424}, \frac{18453763328}{60529284729}, -\frac{19040799232}{6576074649} \Big\}, $$
$$ \Big\{ -\frac{14}{45}, \frac{135}{56}, -\frac{185}{504}, \frac{25432135}{14622776},
\frac{11585718144}{50291423405}, \frac{314271141184}{117352732005} \Big\}, $$
$$ \Big\{ \frac{14}{45}, -\frac{135}{56}, -\frac{832}{315}, \frac{21739328}{125951315},
\frac{197932494375}{570623898632}, -\frac{207609892105}{76457704968} \Big\}, $$
$$ \Big\{ -\frac{14}{45}, \frac{77}{40}, \frac{135}{56}, \frac{203687253}{361681960},
-\frac{5323853454400}{12959750399967}, \frac{4826209930880}{3371383988343} \Big\}, $$
$$ \Big\{ -\frac{7}{17}, -\frac{425}{1008}, \frac{2432}{1071},
\frac{80888528768}{50503742919}, \frac{1661966668042065}{1421147949949456},
\frac{13748985346416705}{5799449383741456} \Big\}. $$

}

\section{Curves with the rank $8$}

The examples of rational Diophantine sextuples found by Gibbs were used in
\cite{D-irr} and \cite{D-mw} to find examples of elliptic curves
of the form
\begin{equation} \label{81}
y^2= (ax+1)(bx+1)(cx+1)(dx+1),
\end{equation}
where $\{a,b,c,d\}$ is a Diophantine quadruple,
and
\begin{equation} \label{82}
y^2= (ax+1)(bx+1)(cx+1),
\end{equation}
where $\{a,b,c\}$ is a Diophantine triple, with relatively large rank.
In both cases, examples with rank equal to $8$
were found. Using the examples from the previous section, i.e. taking
$\{a,b,c,d\}$ and $\{a,b,c\}$ to be subquadruples and subtriples of Diophantine
sextuples, we can find more examples
with the same property. Indeed, we have found by MWRANK that the curve (\ref{81}) has rank $8$ for
$$ \{a,b,c,d\} = \left\{\frac{385}{96}, \frac{1280}{243},
\frac{250705}{44376}, -\frac{25415}{161376} \right\}, $$
while the curve (\ref{82}) has rank $8$ for $\{a,b,c\}$ equal to
$$ \left\{ -\frac{1530}{361}, \frac{2088}{9245}, \frac{399245}{2889816} \right\},  \,\,\,\,
 \left\{ \frac{8695}{1656}, \frac{54648}{22201}, \frac{46288935}{9481336} \right\}, $$
$$ \left\{ \frac{8695}{1656}, -\frac{11270}{62001}, \frac{46288935}{9481336} \right\}, \,\,\,\,
 \left\{ \frac{21}{22}, \frac{1092049959}{1018087688}, -\frac{778578801}{1587999368} \right\}, $$
$$ \left\{ \frac{96305}{158404}, \frac{23144992}{59202405}, -\frac{31157568}{20220605} \right\}, \,\,\,\,
 \left\{ \frac{269654}{113569}, \frac{11572496}{19969047}, -\frac{15578784}{44488087} \right\}. $$
The ranks have been computed unconditionally, except for the last two curves where MWRANK gives
that the rank is equal to $7$ or $8$, while the Parity Conjeture gives that the rank is even.

\begin{thebibliography}{99}

\small{
\bibitem{A-H-S}
J. Arkin, V. E. Hoggatt and E. G. Strauss,
On Euler's solution of a problem of Diophantus, Fibonacci Quart.
{\bf 17} (1979), 333--339.

\bibitem{pari}
C. Batut, D. Bernardi, H. Cohen and M. Olivier, GP/PARI, Universit\'e
Bordeaux I, 1994.

\bibitem{C-H-M}
L. Caporaso, J. Harris and B. Mazur, Uniformity of
rational points, J. Amer. Math. Soc. {\bf 10} (1997), 1--35.

\bibitem{mwrank}
J. E. Cremona, Algorithms for Modular Elliptic Curves,
Cambridge Univ. Press, 1997.

\bibitem{Dio}
Diophantus of Alexandria, Arithmetics and the Book of
Polygonal Numbers, (I. G. Bashmakova, Ed.),
Nauka, Moscow, 1974 (in Russian).

\bibitem{D-acta2}
A. Dujella, On Diophantine quintuples, Acta Arith. {\bf 81} (1997),
69--79.

\bibitem{D-rim}
A. Dujella, Diophantine m-tuples and elliptic curves,
J. Th\'eor. Nombres Bordeaux {\bf 13} (2001), 111--124.

\bibitem{D-irr}
A. Dujella, Irregular Diophantine m-tuples and high-rank elliptic curves,
Proc. Japan Acad. Ser. A Math. Sci. {\bf 76} (2000), 66--67.

\bibitem{D-crelle}
A. Dujella, There are only finitely many Diophantine quintuples, J. Reine Angew. Math.
{\bf 566} (2004), 183--214.

\bibitem{D-mw}
A. Dujella, On Mordell-Weil groups of elliptic curves induced by Diophantine
triples, Glas. Mat. Ser. III {\bf 42} (2007), 3--18.

\bibitem{Fuj-reg}
Y. Fujita, Any Diophantine quintuple contains a regular Diophantine quadruple, preprint.

\bibitem{Gibbs1}
P. Gibbs, Some rational Diophantine sextuples,
Glas. Mat. Ser. III {\bf 41} (2006), 195--203.

\bibitem{Gibbs2}
P. Gibbs, A generalised Stern-Brocot tree from regular Diophantine
quadruples, preprint, {\tt math.NT/9903035}.

\bibitem{Hea}
T. L. Heath, Diophantus of Alexandria. A Study in the History of Greek Algebra.
Powell's Bookstore, Chicago; Martino Publishing, Mansfield Center, 2003.

\bibitem{P-H-Z}
E. Herrmann, A. Peth\H{o} and H. G. Zimmer,
On Fermat's qudruple equations, Abh. Math. Sem. Univ. Hamburg {\bf 69} (1999), 283--291.

}
\end{thebibliography}

\medskip

{\footnotesize
Department of Mathematics, University of Zagreb,

 Bijeni\v{c}ka cesta 30, 10000 Zagreb, Croatia

{\em E-mail address}: {\tt duje@math.hr} }

\end{document}
