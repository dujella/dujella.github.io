\documentclass[11pt]{report}
\topmargin-10mm \textheight=23cm \textwidth=15cm
\oddsidemargin=0.5cm \evensidemargin=0.5cm

%\pagestyle{empty}
%\input{cro-math}
\usepackage[croatian]{babel}

\begin{document}

\begin{center}
{\Large Andrej Dujella - \v{Z}ivotopis} \vspace{3mm}
\end{center}

Ro{\dj}en je 21.5.1966.~u Puli. Djetinjstvo je proveo u \v{Z}minju i
Novigradu Dalmatinskom. Osnovnu \v{s}kolu je poha{\dj}ao u Novigradu
i Zadru (Stanovi), a srednju \v{s}kolu u Zadru. U srpnju
1984.~godine sudjelovao je na Me{\dj}unarodnoj matemati\v{c}koj
olimpijadi u Pragu, gdje je osvojio 3.~nagradu.

Studij za dipl.~in\v{z}enjera matematike na
Prirodoslovno-matemati\v{c}kom fakultetu - Mate\-ma\-ti\v{c}kom
odjelu u Zagrebu, upisao je 1985.~godine. Dobitnik je stipendije
Sveu\v{c}ili\v{s}ta u Zagrebu u \v{s}k.~god.~1988/89. Diplomirao je
na smjeru matemati\v{c}ka informatika i statistika 1990.~godine.

Poslijediplomski studij iz znanstvenog podru\v{c}ja matematike upisao
je 1990.~godine. Magistrirao je 1993.~godine s radnjom {\it
Invarijante topolo\v{s}kih konveksnosti}. Voditelj rada bio je
prof.~dr.~Zvonko \v{C}erin. Doktorirao je 1996.~godine s disertacijom
{\it Generalizirani Diofant-Davenportov problem}. Voditelji
disertacije bili su prof.~dr.~Dragutin Svrtan i
prof.~dr.~Dimitrije Ugrin-\v{S}parac.

Od 1990.~godine zaposlen je kao pripravnik-postdiplomand na PMF -
Matemati\v{c}kom odjelu Sveu\v{c}ili\v{s}ta u Zagrebu. Od 1993.~godine radi
na istom Fakultetu kao asistent, od 1997.~kao docent, od 2000.~kao
izvanredni profesor, od 2004.~kao redoviti profesor, a do 2009.~kao
redoviti profesor u trajnom zvanju.

\medskip

Na dodiplomskom studiju dr\v{z}ao je predavanja iz 10 kolegija
(devet na PMF-MO i jedan na FER-u), te
vje\v{z}be iz 15 kolegija. Uveo je tri nova kolegija: {\it Uvod u teoriju
brojeva}, {\it Elementarna teorija brojeva} i {\it Kriptografija}.
Na poslijediplomskom studiju dr\v{z}ao je predavanja iz kolegija
{\it Diofantske aproksimacije}, {\it Teorija brojeva u
kriptografiji}, {\it Diofantske jednad\v{z}be} i {\it Algoritmi za elipti\v{c}ke krivulje}.
Bio je mentor 155 diplomskig radova.
Tako\dj{}er je bio mentor osam
ma\-gi\-star\-skih radova (Alan Filipin: {\it Primjena LLL-algoritma
u rje\v{s}avanju diofantskih jednad\v{z}bi}; Bernadin
Ibrahimpa\v{s}i\'{c}: {\it Matemati\v{c}ke osnove kriptografije
javnog klju\v{c}a}; Danijel Kop\v{c}inovi\'{c}: {\it Problem
diskretnog logaritma za elipti\v{c}ke krivulje}; Petra Tadi\'{c}:
{\it Metode ra\v{c}unanja ranga elipti\v{c}kih krivulja nad ${\bf
Q}(T)$}; Ana Jurasi\'{c}: {\it Diofantske jednad\v{z}be nad
funkcijskim poljima}; Mirela Juki\'c Bokun: {\it Mestreova
polinomijalna metoda za konstrukciju elipti\v{c}kih krivulja velikog
ranga}; Vinko Petri\v{c}evi\'c: {\it Periodski veri\v{z}ni razlomci};
Luka Lasi\'c: {\it Visine u diofantskoj geometriji i posljedice $abc$-slutnje})
i \v{c}etiri doktorske disertacije (Borka Jadrijevi\'{c}:
{\it Dvoparametarska familija Thueovih jednad\v{z}bi \v{c}etvrtog
stupnja}; Zrinka Franu\v{s}i\'{c}: {\it Diofantove \v{c}etvorke u
kvadratnim poljima}; Alan Filipin: {\it Sustavi pellovskih
jednad\v{z}bi i problem pro\v{s}irenja nekih Diofantovih trojki};
Bernadin Ibrahimpa\v{s}i\'c: {\it Primjena veri\v{z}nih razlomaka u
diofantskim aproksimacijama i kriptoanalizi}).
Svi doktoranti imaju uspje\v{s}ene karijere kao sveu\v{c}ili\v{s}ni profesori ili docenti.
Alan Filipin je za zajedni\v{c}ki znanstveni rad koji je proiza\v{s}ao iz njegove disertacije
dobio nagradu Dru\v{s}tva sveu\v{c}ili\v{s}nih nastavnika i drugih znanstvenika u Zagrebu,
za mlade znanstvenike i umjetnike za 2007. godinu.
U tijeku je izrada vi\v{s}e
doktorskih disertacija pod vodstvom Andreja Dujelle (pet njegovih trenutnih doktorskih studenata
ve\'c ima objavljene ili prihva\'cene za objavljivanje \v{c}lanke u
\v{c}asopisima s SCIE liste).

\medskip

\v{C}lan je Hrvatskog matemati\v{c}kog dru\v{s}tva, American Mathematical
Society, The Fibonacci Association i International Association for
Cryptologic Research. Petnaest godina bio je \v{c}lan Dr\v{z}avnog
povjerenstva za matemati\v{c}ka natjecanja.

\newpage

Voditelj je znanstveno-istra\v{z}iva\v{c}kog projekta {\it Diofantske
jednad\v{z}be i elipti\v{c}ke} (\v{s}ifra: 037-0372781-2821), kojeg financira
Ministarstvo znanosti, obrazovanja i \v{s}porta Republike Hrvatske.
Projekt predstavlja nastavak prethodnog
petogodi\v{s}njeg projekta {\it Diofantske jednad\v{z}be} (\v{s}ifra:
0037110). Ta dva projekta spadaju me\dj{}u najuspje\v{s}ije matemati\v{c}ke
projekte u Republici Hrvatskoj. U okviru tih projekta objavljen je ukupno
61 znanstveni rad (od \v{c}ega 34 u CC, a 53 u SCIE \v{c}asopisima), dok je jo\v{s}
11 radova prihva\'ceno za objavljivanje.


Trenutno je voditelj hrvatske strane projekta u tri me{\dj}unarodna
znanstvena projekta, a bio je voditelj jo\v{s} pet takvih projekta:
\begin{itemize}
\item hrvatsko-austrijski projekti {\it Algorithmic solution of Diophantine
equations and applications to cryptography } (2004--2005), {\it Algorithmic solution
of Diophantine equations and applications to cryptography II} (2006--2007),
{\it Diophantine equations and additive representations} (2008--2009) i
{\it Arithmetic-combinatorial problems and applications} (2010--2011) financirani u okviru
znanstveno-tehni\v{c}ke suradnje Hrvatska-Austrija (voditelj
austrijske strane projekata je profesor Robert Tichy s Tehni\v{c}kog
sveu\v{c}ili\v{s}ta u Grazu);
\item hrvatsko-francuski projekti {\it Diophantine equations} (2005--2006)
i {\it Diophantine approximations} (2009--2010)
financirani u okviru Programa integriranih akcija "Cogito" (voditelj francuske strane
projekata je profesor Yann Bugeaud sa Sveu\v{c}ili\v{s}ta Louis Pasteur u Strasbourgu);
\item hrvatsko-ma{\dj}arski projekti
{\it Investigations in number theory and cryptography} (2005--2007) i
{\it Number theory and cryptography} (2009--2011) (voditelj ma{\dj}arske strane projekata je
profesor Attila Peth\H{o} sa Sveu\v{c}ili\v{s}ta u Debrecinu) .
\end{itemize}

\medskip

Suvoditelj je (uz prof.~dr.~sc.~Ivicu Gusi\'ca) poslijediplomskog {\it Seminara za teoriju brojeva
i algebru} (od 2001.~godine).
Seminar ima trenutno 24 aktivna \v{c}lana, a na njemu \v{c}esto gostuju ugledni me\dj{}unarodni eksperti
iz polja teorije brojeva.
Izvr\v{s}ni urednik znanstvenog \v{c}asopisa
{\it Glasnik Matemati\v{c}ki} (od 2002.~godine).
Tijekom njegovog mandata, \v{c}asopis je u\v{s}ao je presti\v{z}ne
baze {\it Reference List Journals of MathSciNet} i {\it Science Citation Index Expanded}.
Od 2003.~do 2006.~godine bio je glavni urednik {\it Hrvatskog matemati\v{c}kog
elektronskog \v{c}asopisa math.e}. Bio je predstojnik {\it Zavoda za
algebru i osnove matematike} 2001--2003 i 2005--2007. Bio je \v{c}lan
{\it Mati\v{c}nog odbora za polje matematika} (2005--2009), a od 2009. godine je
\v{c}lan {\it Podru\v{c}nog znanstvenog vije\'ca za prirodne znanosti}.

\medskip

Sudjelovao je na me{\dj}unarodnim znanstvenim skupovima u Austriji,
\v{C}e\v{s}koj, Francuskoj, Italiji, Japanu, Kanadi, Ma{\dj}arskoj, Njema\v{c}koj, Slova\v{c}koj, Velikoj
Britaniji i Hrvatskoj, na kojima je odr\v{z}ao ukupno 33 priop\'{c}enja (od toga na 14 skupova
kao pozvani predava\v{c}).

U prosincu 1996., te u studenom i prosincu 1998.~bio je na
studijskom boravku na Matemati\v{c}kom Institutu Sveu\v{c}ili\v{s}ta Lajos
Kossuth u Debrecinu, Ma{\dj}arska. U o\v{z}ujku 2001.~bio je gostuju\'{c}i
profesor na Technische Universit\"at u Grazu, Austrija. Tom
prilikom dr\v{z}ao je predavanja na poslijediplomskom
studiju iz kolegija {\it Elliptic Curves and Applications}.
U svibnju 2007.~bio je gostuju\'{c}i
profesor na Faculty of Informatics, University of Debrecen,
te sam na doktorskom
studiju dr\v{z}ao predavanja iz kolegija {\it Algorithmic Aspects of Elliptic Curves}.

Predavanja po pozivu odr\v{z}ao je na sveu\v{c}ili\v{s}tima u Bilbau, Budimpe\v{s}ti,
Debrecinu, Grazu, Leobenu, Osijeku, Saarbruckenu, Splitu i Tokiju.

Bio je suorganizator, te \v{c}lan znanstvenog ili programskog odbora me\dj{}unarodnih
znan\-stve\-nih skupova odr\v{z}anih u Splitu, Osijeku, Klagenfurtu, Debrecinu i Banffu.

\medskip

Glavno podru\v{c}je njegovog znanstvenog rada je teorija brojeva,
posebno diofantske jednad\v{z}be. Glavni znanstveni doprinos
mu je vezan uz teoriju Diofantovih $m$-torki, tj. skupova prirodnih
brojeva sa svojstvom da umno\v{z}ak svaka dva njihova
razli\v{c}ita \v{c}lana uve\'{c}an za $1$ daje potpun kvadrat.
Na primjer, skup \{1,3,8,120\} je Diofantova \v{c}etvorka. Ovu
\v{c}etvorku je prona\v{s}ao Fermat. Baker i Davenport su
1969.~godine dokazali da se ova \v{c}etvorka ne mo\v{z}e
nadopuniti do Diofantove petorke, te od tog vremena datira slutnja
da ne postoji niti jedna Diofantova petorka.
Ova slutnja smatra se jednim od va\v{z}nijih nerije\v{s}enih problema iz
polja diofantskih jednad\v{z}bi (vidi npr. [{\sc M.~Waldschmidt},
Open Diophantine problems, Moscow Math. J. 4 (2004), 245-305.],
[R.~K.~Guy, Unsolved Problems in Number Theory, 3rd edition, Springer-Verlag, New York, 2004, Section D29]).
Kori\v{s}tenjem veze ovog problema s elipti\v{c}kim krivuljama,
lako je zaklju\v{c}iti da ne postoji beskona\v{c}an skup s gornjim
svojstvom. Me{\dj}utim, tek je u radu A.~Dujelle iz 2001.~godine,
prvi put dana apsolutna granica za veli\v{c}inu Diofantovih
$m$-torki. Naime, dokazano je da ne postoji Diofantova devetorka.
Nakon toga je uspio dokazati i puno ja\v{c}i rezultat, da
ne postoji Diofantova \v{s}estorka, te da Diofantovih petorki ima
samo kona\v{c}no mnogo. Taj je rezultat objavljen 2004.~u
presti\v{z}nom \v{c}asopisu {\it Journal f\"ur die Reine und Angewandte Mathematik}.

Prou\v{c}avao je i razna poop\'{c}enja Diofantovih $m$-torki (npr.
analogan problem u skupu racionalnih brojeva, Gaussovih
cijelih brojeva i polinoma; zamjena kvadrata
vi\v{s}im potencijama), te njihove veze s elipti\v{c}kim
krivuljama, Pellovim jednad\v{z}bama i Fibonaccijevim brojevima).
Prou\v{c}avao je i skupove
prirodnih brojeva u kojima produkt svaka dva razli\v{c}ita elementa
uve\'{c}an za fiksni cijeli broj $n$ daje potpun kvadrat. U zajedni\v{c}kom
radu s F.~Lucom, obajavljenim u uglednom \v{c}asopisu
{\it International Mathematics Research Notices}, dobivena je apsolutna
ocjena za veli\v{c}inu takvih skupova u slu\v{c}aju kada je $n$ prost broj.

U zajedni\v{c}kom radu s C.~Fuchsom iz 2005.~godine,
objavljenom u presti\v{z}nom \v{c}asopisu {\it Journal of the London Mathematical Society},
rije\v{s}io je jedan problem
koji je postavio poznati matemati\v{c}ar Euler.
Naime, Euler je pronasao \v{c}etiri racionalna broja
$\frac{5}{2}, \frac{9}{56}, \frac{65}{224}, \frac{9}{224}$ sa svojstvom da produkt bilo
koja dva broja medu njima, uve\'{c}an za sumu ta dva broja, daje potpun kvadrat,
te je postavio pitanje postoje li \v{c}etiri prirodna broja s istim svojstvom.
U spomenutom radu, dokazano je da takva \v{c}etiri prirodna broja ne postoje.
U zajedni\v{c}kom radu s C.~Fuchsom i A.~Filipinom iz 2007.~godine pokazano je da postoji
samo kona\v{c}no mnogo \v{c}etvorki cijelih brojeva s promatranim svojstvom.

Tako{\dj}er je objavljivao radove vezane uz konstrukciju elipti\v{c}kih krivulja velikog
ranga. Na skupu svih racionalnih to\v{c}aka na elipti\v{c}koj
krivulji mo\v{z}e se uvesti operacija zbrajanja, uz koju taj skup
postaje kona\v{c}no generirana abelova grupa. Jedna od
najpoznatijih slutnji u aritmeti\v{c}koj geometriji jest da broj
generatora (rang) mo\v{z}e biti proizvoljno velik. Ovo pitanje je
donekle povezano i s ocjenom sigurnosti kriptosustava zasnovanih
na elipti\v{c}kim krivuljama. Do danas nije poznata niti jedna
elipti\v{c}ka krivulja ranga ve\'{c}eg od 28.  Krivulja ranga 15, koju je prona\v{s}ao
A. Dujella 2002.~godine,
bila je tada elipti\v{c}ka krivulja najve\'{c}eg ranga za koju je rang egzaktno izra\v{c}unan
(a ne samo dana donja ograda za rang).

Pored ove dvije osnovne grupe problema, objavljivao je i
znanstvene radove vezane za Thueove jednad\v{z}be i
nejednad\v{z}be; nerastavljivost polinoma i diofantske jednad\v{z}be
povezane s rekurzivnim nizovima polinoma; vezu veri\v{z}nih razlomaka i
Newtonove metode za ra\v{c}unanje korijena; kombinatorne
dokaze identiteta povezanih s Fibonaccijevim brojevima;
primjenu diofantskih aproksimacija u kriptoanalizi RSA kriptosustava.

\medskip

Do sada je objavio 63
znanstvenih radova, a ima jo\v{s} 3 rada prihva\'cena za objavljivanje.
Od toga je 35 samostalnih radova,
dok su ostali u koautorstvu sa sljede\'cim matemati\v{c}arima:
Y.~Bugeaud, M.~N.~Deshpande, A.~Filipin, Z.~Franu\v{s}i\'c, C.~Fuchs, I.~Gusi\'c,
B.~Ibrahimpa\v{s}i\'c, B.~Jadrijevi\'c, A.~S.~Janfada, A.~Jurasi\'c, F.~Luca,
M.~Mignotte, A.~Peth\H{o}, V.~Petri\v{c}evi\'c, A.~M.~S.~Ramasamy, S.Salami, I.~Soldo,
P.~Tadi\'c, R.~F.~Tichy, P.~G.~Walsh.
Od ovih 66 radova, 50 radova je objavljeno u \v{c}asopisima koje indeksira
Science Citation Index Expanded, 32 u onima koje indeksira Current Contents,
a 53 u \v{c}asopisima koji su u Mathematical Reviews
referirani ``cover-to-cover'' (MRc-c).

Ima i 13 objavljenih stru\v{c}nih radova. Autor je knjiga {\it
Matemati\v{c}ka natjecanja u\v{c}enika srednjih \v{s}kola}
(koautori: M.~Bombardelli i S.~Slijep\v{c}evi\'c), {\it
Fibonaccijevi brojevi} i sveu\v{c}ili\v{s}nog ud\v{z}benika {\it Kriptografija}
(koautor: M.~Mareti\'c).

\medskip

Na dan 16.~1.~2010.~imao je ukupno
395 citata u SCI (ISI Web of Science),
288 citata u MR Citation Database, te 277 citata u Scopus Citation Tracker.

Najcitiraniji \v{c}lanci su mu:

\begin{enumerate}
\item A. Dujella and A. Peth\H{o}, {\it A generalization of a theorem of Baker and Davenport},
Quart. J. Math. Oxford Ser. (2) {\bf 49} (1998), 291-306. ({\bf 35} citata u MathSciNet)
\item A. Dujella, {\it There are only finitely many Diophantine quintuples},
J. Reine Angew. Math. {\bf 566} (2004), 183-214. ({\bf 31} citat u MathSciNet)
\item A. Dujella, {\it An absolute bound for the size of Diophantine $m$-tuples},
J. Number Theory {\bf 89} (2001), 126-150. ({\bf 23} citata u MathSciNet)
\item A. Dujella, {\it The problem of the extension of a parametric family of Diophantine triples},
Publ. Math. Debrecen, {\bf 51} (1997), 311-322. ({\bf 16} citata u MathSciNet)
\item A. Dujella, {\it On the size of Diophantine $m$-tuples},
Math. Proc. Cambridge Philos. Soc. {\bf 132} (2002), 23-33. ({\bf 13} citata u MathSciNet)
\item A. Dujella, {\it A proof of the Hoggatt-Bergum conjecture},
Proc. Amer. Math. Soc. {\bf 127} (1999), 1999-2005. ({\bf 12} citata u MathSciNet)
\item A. Dujella, {\it On Diophantine quintuples}, Acta Arith. {\bf 81} (1997), 69-79.
({\bf 12} citata u MathSciNet)
\item A. Dujella, {\it Complete solution of a family of simultaneous Pellian equations},
Acta Math. Inform. Univ. Ostraviensis 6 (1998), 59-67. ({\bf 11} citata u MathSciNet)
\item A. Dujella, {\it Generalization of a problem of Diophantus}, Acta Arith. {\bf 65} (1993), 15-27.
({\bf 11} citata u MathSciNet)
\item Y. Bugeaud and A. Dujella, {\it On a problem of Diophantus for higher powers},
Math. Proc. Cambridge Philos. Soc. {\bf 135} (2003), 1-10. ({\bf 10} citata u MathSciNet)
\end{enumerate}

Radovi su mu citirani u \v{c}lancima ili preprintima preko 140
matemati\v{c}ara i kriptografa.
Tako\dj{}er su citirani u sljede\'{c}im knjigama:

\medskip
{\footnotesize

\noindent{\sc A.~Baker}, {\sc G.~W\"ustholz}, Logarithmic Forms and Diophantine Geometry,
Cambridge University Press, Cambridge, 2008.

\noindent{\sc G.~Everest}, {\sc A.~van der Poorten}, {\sc I.~Shparlinski}
and {\sc T.~Ward}, Recurrence Sequences, AMS Surveys and
Monographs, Vol. 104, 2003.

\noindent{\sc M.~J.~Hinek}, Cryptanalysis of RSA and Its Variants, CRC Press, Boca Raton, 2009.

\noindent{\sc P.-C.~Hu}, {\sc C.-C.~Yang}, Value Distribution Theory Related to Number Theory, Birkh\"auser, Basel, 2006.

\noindent{\sc R.~K.~Guy}, Unsolved Problems in Number Theory,
3rd edition, Springer-Verlag, New York, 2004.

\noindent{\sc J.-M.~de Koninck},
Those Fascinating Numbers, American Mathematical Society, Providence, 2009.

\noindent{\sc A.~Peth\H{o}}, Algebraische Algorithmen, Vieweg,
Braunschweig, 1999.

\noindent{\sc I.~E.~Shparlinski}, Finite Fields: Theory and Computation.
The Meeting Point of Number Theory, Computer Science, Coding Theory and Cryptography, Kluwer, Dordrecht, 1999.

\noindent{\sc J.~Steuding}, Diophantine Analysis, Chapman \& Hall/CRC, Boca Raton, 2005.

\noindent{\sc E.~W.~Weisstein}, CRC Concise Encyclopedia of
Mathematics, Chapman \& Hall / CRC, Boca Raton, 1999.
}
\medskip

Linkovi na njegove web stranice ({\tt http://web.math.hr/\~{}duje/tors/tors.html}
i \linebreak {\tt http://web.math.hr/\~{}duje/tors/rankhist.html})
s podatcima o elip\-ti\-\v{c}kim krivuljama velikog ranga
nalaze se na reprezentativnim web stranicama o elipti\v{c}kim krivuljama
i algoritamskoj teoriji brojeva, \v{c}iji su autori
ugledni matemati\v{c}ari {\sc
J.~Cremona}, {\sc N.~Elkies}, {\sc F.~Lemmermeyer}, {\sc
K.~Matthews} i {\sc J.~H.~Silverman}. Stranice su citirane tako{\dj}er i u knjigama

\medskip
{\footnotesize
\noindent{\sc Yu.~I.~Manin} and {\sc A.~A.~Panchishkin}, Introduction to Modern Number Theory,
Springer-Verlag, New York, 2005.

\noindent{\sc J.~S.~Milne}, Elliptic Curves, BookSurge Publishers, 2006.
}
\medskip

Recenzirao je preko 60 \v{c}lanaka u 40 me\dj{}unarodnih znanstvenih \v{c}asopisa (izme\dj{}u ostalih
za \v{c}asopise {\it Journal f\"ur die Reine und Angewandte Mathematik},
{\it Mathematical Proceedings of the Cambridge Philosophical Society},
{\it Mathematics of Computation},
{\it Experimental Mathematics},
{\it Moscow Mathematical Journal}, {\it Journal of Combinatorial Theory, Series A},
{\it Acta Arithmetica}, {\it Journal of Number Theory},
{\it IEEE Transactions on Information Theory}).
Tako\dj{}er je bio recenzent za \v{c}etiri sveu\v{c}ili\v{s}na ud\v{z}benika.

\medskip

Dobitnik je {\it Nagrade Hrvatskog matemati\v{c}kog dru\v{s}tva} mladom
znanstveniku za istaknuti znanstveni doprinos u matematici
2000.~godine, {\it Nagrade Hrvatske akademije znanosti i umjetnosti}
za najvi\v{s}a znanstvena i umjetni\v{c}ka dostignu\'{c}a u Republici Hrvatskoj za 2003.
godinu za podru\v{c}je prirodnih znanosti i matematike, te {\it Dr\v{z}avne nagrade za znanost}
za 2006. godinu.

\medskip

U ljeto 1995.~godine, kao pripadnik 112.~brigade Hrvatske vojske,
sudjelovao je u akciji Oluja.

\medskip

O\v{z}enjen je. Supruga mu se zove Valentina. Imaju troje djece:
Martu (ro{\dj}ena 1994.), Dominika (ro{\dj}en 1999.) i Jelenu
(ro{\dj}ena 2001.).

\end{document}
